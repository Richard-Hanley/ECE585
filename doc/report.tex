\documentclass[11pt,letterpaper,final]{article}
\usepackage[latin1]{inputenc}
\usepackage{amsmath}
\usepackage{amsfonts}
\usepackage{amssymb}
\usepackage{appendix} % Adds additional appendix support
\usepackage{graphicx} % Adds support for graphics
\usepackage{listings} % Adds support for syntax highlighting.
\usepackage{verbatim} % For shell output
\usepackage{fancyvrb} % Gives me some more options on verbatiminput
\usepackage{url} % For wikipedia

\usepackage{float}


\author{Spenser Gilliland \& Richard Hanley}
\title{Project 3: Simulation of CPU Cache and \\ Memory Subsystem}

\begin{document}

\maketitle
\bigskip
\bigskip
\bigskip

\section{ Abstract }
\paragraph{}
This report depicts the design process as well as simulation results of a memory hierarchy which includes a 512 byte instruction cache, 256 byte data cache and 2048 byte main memory.  The project was successful and the collected data is presented in section \ref{Results}.  This design is notable because the memory hierarchy was not only simulated but synthesized for an FPGA.  The results from synthesis are included in section \ref{Results:Synthesis}.  Furthermore, an assembler and linker were created to allow us the opportunity to experiment with different types of programs inorder to better assess the effectiveness of the memory hierarchy.

\pagebreak
\section{ Introduction }
\paragraph{}
This project presented an interesting opportunity to verify and evaluate the effectiveness of cache schemes on the performance of a computer program.  The programs that were evaluated were designed to find both worse case and best case performance as well as to show the effect of the locality of memory on the overall effectiveness of the processor. In addtion, the design and implementation of various memory hierarchy elements are shown.

\section{ Background }

\paragraph{Register File}

\paragraph{Memory}
In order to understand how a memory hierarchy works, one must first understand how memory works.  In the large percentage of memory elements the access time of the memory element is much larger than the access time of local registers.  This is in order to reduce the costs of the external memory elements.  The memory elements are usually controlled by a memory controller which enforces the constraints on the memory elements and allows a user to use them using the basic address, wr, and data lines

\paragraph{Cache}



In a memory hierarchy there is three components: the memory, the bus, and the cache


Both the eight bit CPU and the 32-bit CPU have the same basic design and instruction set. The purpose of a CPU is to implement a set of
instructions and provide a mechanism to execute a series of instructions. 
The process by which a processor executes instructions is called the fetch and
execute cycle.  This simply means that the processor executes an
instruction and then based on that instruction fetches the next instruction. 
This CPU has the following instruction set: \cite{instructions}

\begin{center} 
\begin{small}
\begin{tabular}{ | c | c | c | } 
\hline ASM Code      & Machine Code                             & Comments      
      \\
\hline sll @s @a @b  & \verb|0000 addr_s[3:0] addr_a[3:0] addr_b[3:0]| &
\verb|@s=@a<<@b[5:0]       |\\ 
\hline srl @s @a @b  & \verb|0001 addr_s[3:0] addr_a[3:0] addr_b[3:0]| &
\verb|@s=@a>>@b[5:0]       |\\ 
\hline add @s @a @b  & \verb|0010 addr_s[3:0] addr_a[3:0] addr_b[3:0]| &
\verb|@s=@a+@b             |\\ 
\hline sub @s @a @b  & \verb|0011 addr_s[3:0] addr_a[3:0] addr_b[3:0]| &
\verb|@s=@a-@b             |\\ 
\hline and @s @a @b  & \verb|0100 addr_s[3:0] addr_a[3:0] addr_b[3:0]| &
\verb|@s=@a&@b             |\\ 
\hline xnor @s @a @b & \verb|0101 addr_s[3:0] addr_a[3:0] addr_b[3:0]| &
\verb|@s=~(@a^@b)          |\\ 
\hline loadi @s, imm & \verb|1010 addr_s[3:0] imm[7:0]               | &
\verb|@s=imm               |\\ 
\hline jz @a, bpc    & \verb|1000 bpc[7:4] addr_a[3:0] bpc[3:0]      | &
\verb|jump to bpc if @a==0 |\\ 
\hline jnz @a, bpc   & \verb|1001 bpc[7:4] addr_a[3:0] bpc[3:0]      | &
\verb|jump to bpc if @a!=0 |\\ 
\hline stop          & \verb|1111 xxxx xxxx xxxx                     | &
\verb|stop the program     |\\ 
\hline
\end{tabular}
\end{small}
\end{center}
\paragraph{}

During development, two programs were created.  One program
computes the product of 254941 (partial ID number) and 4111.  The other computes
the integer quotient and remainder of $ 12 / 5 $.  These programs are
implemented in assembly code and then assembled to machine code by the assembler
in appendix \ref{assembler.c}.

In order to implement the instructions listed above, the system presented in
figure 
\ref{fig:overview} was developed.  There are four main units to this design: 
the Memories, the Program Counter, the ALU, and the Control Logic.  The section
outlined in the CPU box of figure \ref{fig:overview} is the portion of this circuit that will be implemented 
in silicon, and the section outlined in the stimulus box of figure \ref{fig:overview} is the test harness.
  
\begin{figure}[ht]
  \centering
  %\includegraphics[width=4in]{overview}
  \caption{An eight bit CPU with test bench \cite{instructions}}
  \label{fig:overview}
\end{figure}


\pagebreak{}
\paragraph{ Memory }
This design implements a Harvard architecture \cite{wpedia:harvardarch}.  In a
Harvard architecture, the system has two main memories.  One memory holds
the instructions to execute, and the other holds the data. In our
case, the external ROM is the instruction memory, and the internal RAM is the
data memory.  

The internal RAM is a dual-port ram with two read interfaces and one write
interface.  As an address enters the ram, it is decoded from binary encoding to
one hot encoding.  This one hot encoded address controls a tristate which acts
like a multiplexer between all of the lines of memory.  The memory itself is 16 lines wide with 32 flip flops for line per line (eight flip flops per
line in the case of the eight bit CPU.) When the address of the needed memory
line is present, the data port will present the value at that address of memory.

The number of lines in the memory stays the same between the eight bit and 32-bit CPU in order to maintain backward compatibility with the instruction set.  The instruction set only allows addresses of five bits.  Therefore, the maximum number of memory lines is $2^5 = 16$.

\paragraph{ Program Counter }
In all CPU designs, there exists a need for a functional unit that determines the
instruction which should be executed during a given clock cycle. The program counter
is a specialized accumulator that implements this functionality.  The program counter
usually increments at each rising edge of the clock, thereby executing the
next instruction in the program.  However if the instruction happens to be a jump
instruction such as ``jz``, or ``jnz`` then, the address in the ''bpc'' portion
of this instruction is conditionally loaded into the counter given that the data
in address A (from the instruction documentation) is equal to zero or
not equal to zero, respectively.  Additionally, the ''stop'' instruction is a special instruction
which stops the program counter from incrementing.
\paragraph{ ALU }
An arithmetic logic unit or ALU is a computing element that performs the
various mathematical operations in a CPU.  Specifically, this ALU implements the
``add``, ``sub``, ``sll``, ``srl'', ''and'', and ''xnor`` operations.  These
operations take the value of two memory addresses and compute a result which is
present at the output of the ALU.  

The ALU is constantly computing the results of each of these operations, and a
multiplexer is used to choose which answer should be propagated to the output. 
The multiplexer is controlled by the three least significant bits of the instruction
code. Because of the similarities between subtraction and addition, the design
uses a single adder for both addition and subtraction.  This is accomplished by
including a \verb|sub| signal which inverts the value in \verb|port b|, and sets the carry in
bit of the adder high when the least significant bit of the instruction code is
high.  The remaining operations are each implemented with a single possible
address and directly compute the needed result.

\paragraph{ Control Logic }
The control logic is a block of combinational logic that decodes the
instruction and sends the needed information to each of the other components in
the system.  This is done through a combination of basic gates.

\section{ 32-bit CPU Implementation }
\paragraph{}
In this project, an eight-bit CPU is transformed into a 32-bit CPU. This new CPU
shares the same instruction set as the eight-bit CPU but allows the user to 
perform operations on 32-bit data.  Therefore, this CPU just extends the data 
path of the CPU to 32 bits wide.

The data path consists of the ALU, Data Memory, and some extraneous logic.  The 
process of extending the data path is fairly simple, consisting of solely 
adjusting the size of each of these elements.  There is a distinct pattern that 
arises when enlarging the data path. 

The pattern is to increase the number of elements and the width of the input/output 
ports. An example is below.
\lstset{language=verilog, breaklines=true, basicstyle=\scriptsize, escapeinside={\%*}{*)}}
\begin{center}
\begin{lstlisting}
//module dff_8(d, clk, q); %* \begin{footnotesize} \textbf{[1]} \end{footnotesize} *)
module dff_32(d, clk, q);
   //output [31:0] q;   %* \begin{footnotesize} \textbf{[1]} \end{footnotesize} *)
   output [31:0] q;
   //input [31:0]  d;   %* \begin{footnotesize} \textbf{[1]} \end{footnotesize} *)
   input [31:0]  d;
   input 	 clk;	
   
   dff d0(d[0], clk, q[0]);
   dff d1(d[1], clk, q[1]);
   %* $ \quad \quad \quad \vdots $ *) %* \begin{footnotesize} \textbf{[2]} \end{footnotesize} *) 
   dff d30(d[30], clk, q[30]);
   dff d31(d[31], clk, q[31]);
endmodule // dff_32
\end{lstlisting}
\end{center}

In the above code example, the \verb|dff_8| is transformed into a \verb|dff_32|
by:
\begin{enumerate}
  \item Increasing the width of the input and output ports.
  \item Quadrupling the number of D flip flops.
\end{enumerate}

This transform is used throughout the code as a way to make a 32-bit item out of
an eight-bit item.  

There is only one exception to this rule.  In the \verb|shift_left_logic_32| module and 
\verb|shift_right_logic_32| module, the change needed is using the five least significant 
bits of the operand instead of only the three least significant bits.  This 
is because $ \log_{2}(32) = 5 $, and therefore, it takes only the five least significant
bits of the operand to express the maximum shift.

\section{ Architectural Exploration of Adders }
\label{Adders}
\paragraph{}
As mentioned in the introduction,  the carry chain of the carry ripple adder was 
the performance bottleneck of the initial eight-bit design and it will only be 
exacerbated in the 32-bit design.  This is a common problem in the design of 
adders; therefore, several designs have been created which reduce this 
performance bottleneck.  

In this exercise, a valency four carry select adder is instantiated in place of the 
original carry ripple adder.  The carry select adder is a type of adder that 
pre-computes the carry out for the case of either a carry in or no carry 
in.  It multiplexes the result using the actual carry in for the select of the 
multiplexer (see figure \ref{fig:carry_select}.)  

\begin{figure}[ht]
  \centering
  %\includegraphics[width=6in]{carry_select_adder}
  \caption{Carry select adder from \cite{cmosvlsi}}
  \label{fig:carry_select}
\end{figure}

The carry select has an ideal delay of 
\begin{align}
t_{select} = t_{pg} + [ n + (k-2)] t_{A0} + t_{mux}
\end{align}
and a carry ripple adder as an ideal delay of 
\begin{align}
t_{ripple} = t_{pg} + (N-1) t_{A0} + t_{xor} .
\end{align}
subtracting these two equations it is found that the delay of a carry select 
adder in comparison to the carry ripple adder is
\begin{align}
t_{select} - t_{ripple} = ([ n + (k-2)] - (N-1))  t_{A0} + t_{mux} - t_{xor}
\end{align}
for our design this simplifies to 
\begin{align}
t_{select} - t_{ripple} &= ([8 + (4-2)] - (32-1)) t_{A0} + t_{mux} - t_{xor} \\
t_{select} - t_{ripple} &= -21 t_{A0} + t_{mux} - t_{xor}
\end{align}
Given that $ t_{mux} = t_{xor} $ \footnote{note neccessarily true but an okay approximation}, 
this indicates that the carry select adder is ideally faster by 21 single bit carry 
delays or an maximum decrease in delay of approximately $ 21 / 31 \approx 68\% $ \footnote{compute by ignore the effects of $ t_{pg} $, $ t_{mux} $, and $ t_{xor} $}.

The actual implementation results are listed in appendix \ref{s1/timing.rep} and 
appendix \ref{s2/timing.rep}.  From these results, the maximum frequency for the 
carry ripple adder is
\begin{align}
\frac{1000}{librarysetuptime + dataarrivaltime} = \frac{1000}{.36 + 17.57} = 55 MHz.
\end{align}
Additionally, the maximum frequency for the carry select adder is 
\begin{align}
\frac{1000}{librarysetuptime + dataarrivaltime} = \frac{1000}{.36 + 10.97} = 88 MHz.
\end{align}
Because $dataarrivaltime$ is made up of both the propagation delay through the 
carry ripple chain and the delay through the memory system, it is only an 
approximation to $t_{select}$ and $t_{ripple}$.  But using these numbers, the carry 
select adder decreases the critical path delay by $ 10.97 / 17.57  = 61\% $.

Despite these advancements, the critical path is still the path through the 
carry chain.  In order to increase the max operating frequency, a new 
adder design will need to be implemented.  The implementation could take the 
form of any of the adders listed on page 449 of \cite{cmosvlsi}.  

\section{ Division Support }
\paragraph{}
Division support was added to the CPU by implementing the following algorithm in assembly code:

To test and evaluate this algorithm, a prototype of the algorithm was created in 
the C programming language and is available in appendix \ref{s3/proto-div/main.c}.
This algorithm was then rewritten in assembly and can be seen in appendix \ref{s3/code.asm}.

All of the functions in this algorithm are available in the CPU except for the 
''less than or equal to'' comparison.  To implement this algorithm, a comparator was 
added to the ALU. The implemented ''less than or equal to'' comparator reuses the 
existing adder design to implement its comparison functions. It is based on the 
design on page 462 of \cite{cmosvlsi} as shown below.
\begin{figure}[H]
  \centering
  %\includegraphics[width=2.5in]{comparator.png}
  \label{fig:comparator}
  \caption{''Less than or equal to'' comparator from  \cite{cmosvlsi}}
\end{figure}

The verilog implementation is below:


\begin{lstlisting}
module lteq_32(lteq, s, co);
   
   output [31:0] lteq;
   input  [31:0] s;
   input         co;
   
   wire z;
   
   assign lteq[31:1] = 31'b0;
   
   not notc(nco, co);
   or  orc (lteq[0], z, nco); 
   
   zero_detect_32 zero(z, s);

endmodule // lteq_32 
\end{lstlisting}

In order for the \verb|lteq_32| module to work correctly, the \verb|addsub_32| 
circuit must be performing a subtraction operation at the time of the comparison.  
To guarantee this behavior, the \verb|lteq_32| module is place in the op code 
seven position of the ALU and the instructions prototype is as follows.

\begin{center} 
\begin{small}
\begin{tabular}{ | c | c | c | } 
\hline ASM Code      & Machine Code                                     & Comments           \\
\hline lteq @s @a @b & \verb| 0111 addr_s[3:0] addr_a[3:0] addr_b[3:0]| & \verb|@s=@a <= @b| \\
\hline 
\end{tabular}
\end{small}
\end{center}

The new ALU can be described by the following diagram.

\begin{figure}[H]
  \centering
  %\includegraphics[width=4in]{ALU2}
  \caption{''Less than or equal to'' comparator added to ALU}
  \label{fig:alu2}
\end{figure}

\section{ Design Validation }
\paragraph{}
In this section, the procedure for evaluating the validity of the circuit is examined.  There are three steps in validating this circuit: behavioral simulation, post synthesis simulation, and post place and route simulation.

The circuit is put through its paces by the code in appendix \ref{s1/code.asm} and \ref{s3/code.asm}.  The first program computes the product of the values in registers two and three, and stores it in register six.  Then the program, subtracts the product by register six so that it eventually returns to zero.  The second program performs integer division on the values in registers two and three. It returns the quotient in register five and the remainder in register six.  It is designed to test the operation of the "less than or equal to" circuit.  

\paragraph{ Behavioral Simulation }
The first step is a behavioral simulation.  This step shows how the circuit is supposed to behave, given that there are no errors in the synthesis or place and route operations.  It can be considered the baseline simulation of the system.  It is important to have a good behavioral simulation before moving forward because each step of the design flow makes it harder to diagnose or fix bugs.

The behavioral simulation was successful in both cases.  The results can be seen in appendix \ref{s1/behaviour}, \ref{s2/behaviour} and \ref{s3/behaviour}.

\paragraph{ Post-synthesis Simulation }
The next step in validation is running a post synthesis simulation.  In this simulation, the circuit is simulated with additional information from the standard cells that will build the circuit.  As a result, propagation delays can be observed in the waveform.  These propagation delays allow a designer to better estimate how the real circuit will respond to a stimulus.  

The post synthesis simulation was successful in both cases. The results can be seen in appendix \ref{s1/postsyn}, \ref{s2/postsyn} and \ref{s3/postsyn}.

\paragraph{ Post-layout Simulation }
The final step in validation is running a post layout simulation.  In this simulation, the circuit is simulated with additional interconnect and buffer delays.  This is the most accurate simulation possible and should reflect very closely as to how the circuit will react to a stimulus in the field.  If the circuit passes post-layout simulation, then the design can be reasonably guaranteed to perform its desired function.

The post layout simulation was successful in both case.  The results can be seen in appendix \ref{s1/postpr}, \ref{s2/postpr} and \ref{s3/postpr}.

\section{ Synthesis Results }
\paragraph{}

For this exercise, the only required performance characteristics is an operating frequency of 20 MHz.  In order to evaluate this circuits performance, there are two quantitative data points that can be used for evaluation and one quantitative result: layout size, maximum frequency, and critical path.  Layout size is the amount of area that the circuit will take up on a die. Maximum frequency is the maximum frequency at which the circuit can be clocked.  The critical path is the longest path through combinational logic.  These results are evaluated for each stage of the design process: Stage 1 - Carry Ripple Adder, Stage 2- Carry Select Adder, Stage 3 - Division Support.

\paragraph{ Layout Size }
The layout size for the stage 1 circuit is $ 1432404 \mu m^2$.  This means that the circuit would fit in a $ 1196.8 \mu m $ by $ 1196.8 \mu m $ square. The layout size was extracted from the cell.rep in appendix \ref{s1/cell.rep}.

The layout size for the stage 2 circuit is $ 1452465 \mu m^2$.  This means that the circuit would fit in a $ 1205.1 \mu m $ by $ 1205.1 \mu m $ square. The layout size was extracted from the cell.rep in appendix \ref{s2/cell.rep}.  This is expected due to the increase in the number of gates needed to build the carry select adder in place of the carry ripple adder.

The layout size for the stage 3 circuit is $ 1473930 \mu m^2$.  This means that the circuit would fit in a $ 1214.0 \mu m $ by $ 1214.0 \mu m $ square. The layout size was extracted from the cell.rep in appendix \ref{s3/cell.rep}.  This is expected because of the additional gates necessary to build the "less than or equal to" comparator.
 
\paragraph{ Maximum Frequency }
The maximum frequency of this circuit is determined by 
\begin{equation}
\frac{1000}{library setup time + data arrival time} 
\end{equation}
where $librarysetuptime$ and $dataarrivaltime$ are extracted from the 
timing.rep file located in appendix \ref{s1/timing.rep}, \ref{s2/timing.rep}, and \ref{s3/timing.rep}.
In our case, this evaluates to 
\begin{align}
\frac{1000}{.36 + 17.57} &= 55.77 MHz &\mbox{ for Stage 1, } \\
\frac{1000}{.36 + 10.98} &= 88.18 MHz &\mbox{ for Stage 2, } \\
\frac{1000}{.36 + 11.41} &= 84.96 MHz &\mbox{ for Stage 3. } \\
\end{align}

The speeds align with expected values.  The speed-up associated with the change from a carry ripple adder to a carry select is as expected, and the small reduction in speed for stage 3 is due to the addition of the "less than or equal to" comparator.

\paragraph{ Critical Path } 

The critical path for the stage 1 circuit is the path from code[1] through memory port\verb=_=b through the addsub and the carry chain of the adder to the most significant bits of the sum through a couple multiplexers and back to memory through port\verb=_=s.

The critical path for the stage 2 circuit is the path from code[1] through memory port\verb=_=b through the addsub and the carry chain of the carry select adder to the most significant bits of the sum through a couple multiplexers and back to memory through port\verb=_=s.

The critical path for the stage 3 circuit is the path from code[1] through memory port\verb=_=b through the addsub and the carry chain of the addsub to the most significant bits of the sum through the "less than or equal to" comparator.  Through a couple multiplexers and back to memory through port\verb=_=s.

The critical path is extracted from the timing.rep.5 located in appendix \ref{s1/timing.rep.5.final}, \ref{s2/timing.rep.5.final}, and \ref{s3/timing.rep.5.final}.

\section{ Conclusions and Future Work }
\paragraph{}
This report shows the design methods and validation process for extending an
eight bit CPU to a 32-bit CPU, replacing the carry ripple adder in the this CPU
with a carry select adder, and a method for implementing integer division on this
CPU.  After completing the design, it is apparent that additional functionality could be added to this CPU.  The number six opcode could be filled with a "greater than" comparator.  This would provide the programmer with additional functionality from which to build programs.  The performance of the system could be improved by implementing a more advanced adder architecture because the critical path is still through the carry chain of the adder.

\pagebreak{}
\begin{thebibliography}{widestlabel}
\bibitem{progress_report}
  Spenser Gilliland
  Progress Report: Design and Synthesis of Central Processing Units.
  Illinois Institute of Technology, Illinois,
  Fall 2010.
\bibitem{instructions}
  Prof. Jia Wang,
  Final Project: Design and Synthesis of Central Processing Units,
  Illinois Institute of Technology, Illinois,
  Fall 2010.
\bibitem{cmosvlsi}
  Neil H. E. Weste, David Money Harris,
  CMOS VLSI Design: A Circuits and Systems Perspective,
  Addision Wesley, Massachusetts,
  4th Edition,
  2011.
\bibitem{wpedia:harvardarch}
  Wikipedia contributors,
  Harvard Architecture,
  Wikipedia, The Free Encylopedia,
  \verb|http://en.wikipedia.org/wiki/Harvard_architecture| (accessed November
14, 2010)
\end{thebibliography}

\pagebreak{}
\appendixpage
\appendix
\section{ Stage 1: 32-bit CPU }

\subsection{Implementation}
\subsubsection{cpu32.v}
\label{s1/cpu32.v}
\begin{scriptsize}
  %\verbatiminput{s1/cpu32.v}
\end{scriptsize}
\subsubsection{code.asm}
\label{s1/code.asm}
\begin{scriptsize}
  %\verbatiminput{s1/code.asm}
\end{scriptsize}
\subsubsection{code.hex}
\label{s1/code.hex}
%\begin{scriptsize}
  %\verbatiminput{s1/code.hex}
%\end{scriptsize}

\subsection{Verification}
\subsubsection{Behavioral}
\label{s1/behaviour}
%\includegraphics[width=4.7in, trim= 1in 6in .2in 2in, clip=true]{s1/behaviour-part1} \\%\includegraphics[width=4.7in, trim= 1in 6in .2in 2in, clip=true]{s1/behaviour-part2} \\
%\includegraphics[width=4.7in, trim= 1in 6in .2in 2in, clip=true]{s1/behaviour-part3} \\
\begin{scriptsize}
  %\VerbatimInput{s1/behaviour.log}
\end{scriptsize}
\subsubsection{Post Synthesis}
\label{s1/postsyn}
%\includegraphics[width=4.7in, trim= 1in 6in .2in 2in, clip=true]{s1/postsyn-part1} \\
%\includegraphics[width=4.7in, trim= 1in 6in .2in 2in, clip=true]{s1/postsyn-part2} \\
%\includegraphics[width=4.7in, trim= 1in 6in .2in 2in, clip=true]{s1/postsyn-part3} \\
\begin{scriptsize}
  %\VerbatimInput{s1/postsyn.log}
\end{scriptsize}
\subsubsection{Post Place and Route}
\label{s1/postpr}
%\includegraphics[width=4.7in, trim= 1in 6in .2in 2in, clip=true]{s1/postpr-part1} \\
%\includegraphics[width=4.7in, trim= 1in 6in .2in 2in, clip=true]{s1/postpr-part2} \\
%\includegraphics[width=4.7in, trim= 1in 6in .2in 2in, clip=true]{s1/postpr-part3} \\
\begin{scriptsize}
  %\VerbatimInput{s1/postpr.log}
\end{scriptsize}

\subsection{Synthesis}
\subsubsection{cell.rep}
\label{s1/cell.rep}
\begin{scriptsize}
  %\VerbatimInput[firstline=5300]{s1/cell.rep}
\end{scriptsize}
\subsubsection{timing.rep}
\label{s1/timing.rep}
\begin{scriptsize}
  %\VerbatimInput{s1/timing.rep}
\end{scriptsize}
\subsubsection{timing.rep.5.final}
\label{s1/timing.rep.5.final}
\begin{scriptsize}
  %\VerbatimInput[lastline=20]{s1/timing.rep.5.final}
\end{scriptsize}

\section{ Stage 2: 32-bit CPU w/ Carry Skip Adder }

\subsection{Implementation}
\subsubsection{cpu32.v}
\label{s2/cpu32.v}
\begin{scriptsize}
  %\verbatiminput{s2/cpu32.v}
\end{scriptsize}
\subsubsection{code.asm}
\label{s2/code.asm}
\begin{scriptsize}
  %\verbatiminput{s2/code.asm}
\end{scriptsize}
\subsubsection{code.hex}
\label{s2/code.hex}
\begin{scriptsize}
  %\verbatiminput{s2/code.hex}
\end{scriptsize}

\subsection{Verification}
\subsubsection{Behavioral}
\label{s2/behaviour}
%\includegraphics[width=4.7in, trim= 1in 6in .2in 2in, clip=true]{s2/behaviour-part1} \\
%\includegraphics[width=4.7in, trim= 1in 6in .2in 2in, clip=true]{s2/behaviour-part2} \\
%\includegraphics[width=4.7in, trim= 1in 6in .2in 2in, clip=true]{s2/behaviour-part3} \\
\begin{scriptsize}
  %\VerbatimInput{s2/behaviour.log}
\end{scriptsize}
\subsubsection{Post Synthesis}
\label{s2/postsyn}
%\includegraphics[width=4.7in, trim= 1in 6in .2in 2in, clip=true]{s2/postsyn-part1} \\
%\includegraphics[width=4.7in, trim= 1in 6in .2in 2in, clip=true]{s2/postsyn-part2} \\
%\includegraphics[width=4.7in, trim= 1in 6in .2in 2in, clip=true]{s2/postsyn-part3} \\
\begin{scriptsize}
  %\VerbatimInput{s2/postsyn.log}
\end{scriptsize}
\subsubsection{Post Place and Route}
\label{s2/postpr}
%\includegraphics[width=4.7in, trim= 1in 6in .2in 2in, clip=true]{s2/postpr-part1} \\
%\includegraphics[width=4.7in, trim= 1in 6in .2in 2in, clip=true]{s2/postpr-part2} \\
%\includegraphics[width=4.7in, trim= 1in 6in .2in 2in, clip=true]{s2/postpr-part3} \\
\begin{scriptsize}
  %\VerbatimInput{s2/postpr.log}
\end{scriptsize}

\subsection{Synthesis}
\subsubsection{cell.rep}
\label{s2/cell.rep}
\begin{scriptsize}
  %\VerbatimInput[firstline=5300]{s2/cell.rep}
\end{scriptsize}
\subsubsection{timing.rep}
\label{s2/timing.rep}
\begin{scriptsize}
  %\VerbatimInput{s2/timing.rep}
\end{scriptsize}
\subsubsection{timing.rep.5.final}
\label{s2/timing.rep.5.final}
\begin{scriptsize}
  %\VerbatimInput[lastline=20]{s2/timing.rep.5.final}
\end{scriptsize}

\section{ Stage 3: 32-bit CPU w/ Division Support }

\subsection{Implementation}
\subsubsection{cpu32.v}
\label{s3/cpu32.v}
\begin{scriptsize}
  %\verbatiminput{s3/cpu32.v}
\end{scriptsize}
\subsubsection{code.asm}
\label{s3/code.asm}
\begin{scriptsize}
  %\verbatiminput{s3/code.asm}
\end{scriptsize}
\subsubsection{code.hex}
\label{s3/code.hex}
\begin{scriptsize}
  %\verbatiminput{s3/code.hex}
\end{scriptsize}

\subsection{Verification}
\subsubsection{Behavioral}
\label{s3/behaviour}
%\includegraphics[width=4.7in, trim= 1in 6in .2in 2in, clip=true]{s3/behaviour-part1} \\
\begin{scriptsize}
  %\VerbatimInput{s3/behaviour.log}
\end{scriptsize}
\subsubsection{Post Synthesis}
\label{s3/postsyn}
%\includegraphics[width=4.7in, trim= 1in 6in .2in 2in, clip=true]{s3/postsyn-part1} \\

\begin{scriptsize}
  %\VerbatimInput{s3/postsyn.log}
\end{scriptsize}
\subsubsection{Post Place and Route}
\label{s3/postpr}
%\includegraphics[width=4.7in, trim= 1in 6in .2in 2in, clip=true]{s3/postpr-part1} \\
\begin{scriptsize}
  %\VerbatimInput{s3/postpr.log}
\end{scriptsize}

\subsection{Synthesis}
\subsubsection{cell.rep}
\label{s3/cell.rep}
\begin{scriptsize}
  %\VerbatimInput[firstline=5300]{s3/cell.rep}
\end{scriptsize}
\subsubsection{timing.rep}
\label{s3/timing.rep}
\begin{scriptsize}
  %\VerbatimInput{s3/timing.rep}
\end{scriptsize}
\subsubsection{timing.rep.5.final}
\label{s3/timing.rep.5.final}
\begin{scriptsize}
  %\VerbatimInput[lastline=20]{s3/timing.rep.5.final}
\end{scriptsize}

\section{Division Algorithm}
\subsection{proto-div/main.c}


\section{ Utilities }
\subsection{ assembler.c }



\end{document}